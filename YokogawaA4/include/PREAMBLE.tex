\documentclass[12pt]{book}%
%
%%------------------------------
% PREAMBLE
% Standard Header Definitions
% .rf. Jan/2017
%%------------------------------
%
\newcommand{\YOdoLog}[1]{\typeout{*** }\typeout{*** #1 }\typeout{*** }}%
\YOdoLog {PARSING PREAMBLE }%
%
%%-------------------------------------
%%  --  YO(kogawa) VARs and Makros --
%%-------------------------------------
%
\newcommand{\YOincludePath}      {../include}
\newcommand{\YOreqfile}          {Requirements.lis}
%
\newcommand{\YOprintReqCheckList}{}%            % Print Requirements List Table
\newcommand{\YOmakeTOC}       [1]{}%            % Yokogawa Inhaltsverzeichnis erstellen (0=nein, 1=ja)
\newcommand{\YOmakeTitlePage} [1]{}%            % Yokogawa Deckblatt erstellen          (0=nein, 1=ja)
\newcommand{\YOmakeIndex}     [1]{}%            % Yokogawa Index erstellen              (0=nein, 1=ja)
\newcommand{\YOprintIndex}       {}%            % Yokogawa Index drucken
\newcommand{\YOcounter}          {}%            % increment and print counter
\newcommand{\YOsubcounter}       {}%            % increment and print subcounter
\newcommand{\YOsetcounter}    [1]{}%            % set new value to counter
\newcommand{\YOsetsubcounter} [1]{}%            % set new value to subcounter
\newcommand{\YOreqref}        [1]{}%            % reference to Requirement .. wie \ref und \pageref
\newcommand{\YOrequirement}   [3]{}%            % create REQ    (Header, Text, optional: Label)
\newcommand{\YOsubrequirement}[3]{}%            % create SubREQ (Header, Text, optional: Label)
%
\newcommand   {\YOprintReq}   [3]{\printRequirementBox         {#1}{#2}{#3}}%    Req-No., Titel, Text
\newcommand   {\YOprintSubReq}[3]{\printSubRequirementBox      {#1}{#2}{#3}}% SubReq-No., Titel, Text
\renewcommand {\YOprintReq}   [3]{\printRequirementDescription {#1}{#2}{#3}}%    Req-No., Titel, Text
\renewcommand {\YOprintSubReq}[3]{\printRequirementDescription {#1}{#2}{#3}}% SubReq-No., Titel, Text
%
%%-- Yokogawa Logo Definitons ---
%
\newcommand {\YokogawaLogo}{\YOincludePath/YokogawaLogo.png}%
\newcommand {\yokogawa}{\mbox{\includegraphics
                        [trim=0 10pt 0 0 ,width=0.15\columnwidth]{\YokogawaLogo}\ }}%
\newcommand {\Yokogawa}{\yokogawa}%
%
% central footer text
\newcommand{\YOcfooter}{ \directlua{tex.print(math.random())}  }%
%
%%------------------------------
%
\def\pgfsysdriver{pgfsys-pdftex.def}%
%
\usepackage{luatex85}%
\usepackage{luacode}%
%
\usepackage{lmodern}%                           % allows different font sizes
\usepackage{aurical}%
\usepackage[T1]{fontenc}%
\usepackage{fetamont}%
\usepackage[T1]{fontenc}%
%
\usepackage{ifthen}%
\usepackage{blindtext}%
\usepackage{titlesec}%
\usepackage{setspace}%
\usepackage{chngcntr}%                          % counter ctrl settings
\usepackage{verbatim}%
\usepackage{fancyhdr}%                          % Required for custom headers
%
\usepackage{lastpage}%                          % Required to determine the
%                                               % last page for the footer
\usepackage{longtable}
\usepackage{tabularx,colortbl}%
%
\usepackage{extramarks}%                        % Required for headers and
%                                               % footers
\usepackage{xcolor}%
\usepackage{float}%                             % needed to place grafics
%                                               % exactly at .. position
\usepackage[utf8]{luainputenc}%
\usepackage{graphicx}%                          % Required to insert images
%
\usepackage{draftwatermark}%
%
\usepackage{imakeidx}%
\usepackage{tikz}%
%
\usepackage[colorlinks=true,%                   % sollte wegen Nebeneffekten
	        urlcolor=blue,%                     % das letzte Paket in der
	        linkcolor=black]{hyperref}%         % Liste sein ..
%
\renewcommand*{\thechapter}{\arabic{chapter}}%  % modify chapter and counter
%                                               % numbering
\renewcommand*{\thesection}{\arabic{section}}%
%
\setcounter{secnumdepth}{5}%
\counterwithout{figure}{chapter}%               % use counter without chapter#
%
% Margins
\topmargin      =-0.2  in%
\evensidemargin = 0    in%
\oddsidemargin  = 0    in%
\textwidth      = 6.5  in%
\textheight     = 9.0  in%
\headsep        = 0.25 in%
\headheight     = 16pt%
\parskip        =  8pt%
\linespread{1.1}%                               % Line spacing
\author{}%                                      % not used .. using
%                                               % hwkAuthorName instead
%
%\flushbottom                                   % beeinflusst die
\raggedbottom                                   % vertikaleAusrichtung
\hbadness=10000%                                % horizontale Ausrichtung
%
\newcommand{\textbiu}[1]{\emph{\underline{\textbf{#1}}}}% set text bold-italy-underline%
%
\definecolor{ocre}{HTML}{F16723}%               % für Index-Style Definition
%
%----------------------------------------------------------------------------------------
% Set up the header and footer
%----------------------------------------------------------------------------------------
%
\fancyhf{}%                                     % Clear header/footer
\chead{\textbf{Header}}%                        % Centre header
\cfoot{\textbf{Footer}\ \thepage}%              % Centre footer
%
\pagestyle{fancy}%
\makeatletter%
  \let\ps@plain\ps@fancy%                       % plain page style = fancy
\makeatother%
%
\chead{\YOTitle}%                               % Top center header
\rhead{\firstxmark}%                            % Top right header
\lhead{\Yokogawa}%                              % Top left header
%
\lfoot{\vspace{-0.12in}\tiny \today\\
                             \YOAuthorName}%    % Bottom left footer
\cfoot{\YOcfooter}%                             % Bottom center footer
\rfoot{Page\ \thepage\ /\ \pageref{LastPage}}%  % Bottom right footer
%
\renewcommand\headrulewidth{0.4pt}%             % Size of the header rule
\renewcommand\footrulewidth{0.4pt}%             % Size of the footer rule
%
\setlength\parindent{0pt}%                      % Removes indentation from
%
%
%% -------- Contents Makro ----------------------------
%
\renewcommand{\YOmakeTOC}[1]{%
    \setcounter{tocdepth}{5}%
    \ifthenelse{#1=1}{%
        \newpage%
        \renewcommand{\contentsname}{Inhalt}%
        \tableofcontents%
        \newpage%
    }%  end if
    {}% else
}% end \YOmakeTOC
%
%% -------- Title Page Makro  ----------------------------
%
\renewcommand{\YOmakeTitlePage}[1]{%
    \ifthenelse{#1=1}{%
        \maketitle%
    }%  end if
    {}% else
    \pagenumbering{arabic}%
    \setcounter{page}{1}%
}% end \YOmakeTitlePage%
%%
%% -------- Index Generation Makros ----------------------------
%% Makeindex Default Parameter in WinEDT : "%pm" "%N.idx"
%%
\newcommand{\YOmakeIndexPar}{0}%    lokaler Merker
\renewcommand{\YOmakeIndex}[1]{%
    \ifthenelse{#1=1}%
      { % then make 2 indexes : the normal one and one for Requirements
        \makeindex[name=Requirements,title={List of Requirements},options= -s
            \YOincludePath/Requirements.ist, columns=1,intoc=true]%
        \makeindex[options= -s
            \YOincludePath/YokogawaA4.ist, columns=2,intoc=true]%
      }  % end if
      {} % else
    \renewcommand{\YOmakeIndexPar}{#1}%          % Parameter merken für \YOprintindex
}% end \YOmakeIndex

\renewcommand{\YOprintIndex}{%
    \ifthenelse{\YOmakeIndexPar=1}%
      {% then
%       \addcontentsline{toc}{section}{Index}
        \printindex%
        \printindex[Requirements]%               % Requirements index
      } % end if
      {}% else
}% end \YOprintIndex
%%
%% -------- zwei allgemeine counter definieren ----------------------------
%%
\newcounter{YOcounter}%
\newcounter{YOsubcounter}%
%%
%% -- increment counter and reset subcounter, then print counter
\renewcommand{\YOcounter}{\addtocounter{YOcounter}{1}\setcounter{YOsubcounter}{0}\arabic{YOcounter}}%
%% -- increment and print subcounter
\renewcommand{\YOsubcounter}{\addtocounter{YOsubcounter}{1}\arabic{YOsubcounter}}%
%%
%% -- set new values to counter
\renewcommand{\YOsetcounter}[1]{\setcounter{YOcounter}{#1}}%
\renewcommand{\YOsetsubcounter}[1]{\setsubcounter{YOcounter}{#1}}%
%%
%% -------- Makros für REQirements definieren ----------------------------
%% -------- die auch referenzierbar sind ---------------------------------
%%
\renewcommand{\YOreqref}[1]{\reqref{#1}}%    % reference to Requirement .. wie \ref und \pageref
\newcommand{\reqref}[1]{REQ \ref{REQ:#1}}%   % \reqref .. wie \ref und \pageref
%%
%% --------- lokale Definitionen ------------------------------------------
%%
\newcounter{req}%
\newcounter{subreq}[req]%
%
\newcounter{reqsorter}%         % lokale Merker zum Mitzählen der Einträge.
\newcounter{sorter}%            % werden später zur Sortierung im Index gebraucht
%
\newcommand{\tmpIdxMain}{}% % lokaler Merker zum Nachhalten der Hierarchiestufe
%
\renewcommand\thesubreq{\thereq.\arabic{subreq}}%    %lokale Definition
%%
\usepackage{tcolorbox}
\newcommand{\printRequirementBox}[3]{
\begin{tcolorbox}[width=\textwidth,colback={red},title={#1 #2},
                  outer arc=0mm,colupper=white]
       #3
\end{tcolorbox}
}

\newcommand{\printSubRequirementBox}[3]{
\begin{tcolorbox}[width=\textwidth,colback={lightgray},title={#1 #2},
                  colbacktitle=yellow,coltitle=blue]
    #3
\end{tcolorbox}
}

\newcommand{\printRequirementDescription}[3]{
  \begin{description}%                                  % jetzt den Request als "description" ausgeben
      \item [#1] \textbf{#2}

       #3%                                              % den Request Text ausgeben
  \end{description}%
}
%%
%% -------- Neuen Requirement Eintrag erzeugen ----------------------------
% Example:
%%    \requirement{Requirement Header}{Req.text}{ReqLbl}
%%    #1 REQ-Header,  #2 Text, #3 optional Label für \pageref und \reqref
%%
\renewcommand{\YOrequirement}[3]{
              \requirement{#1}{#2}{#3}}%                % Params = Header, Text, optional: Label
%
\newcommand{\requirement}[3]{%
  \refstepcounter{reqsorter}                            % mitzählen
  \refstepcounter{req}
  \renewcommand{\tmpIdxMain}{REQ~\thereq~#1}%           % Eintrag für Index merken
  \index[Requirements]{\thereqsorter@\tmpIdxMain!}%     % Indexeintrag merken
  \YOprintReq{REQ~\thereq}{#1}{#2}
  \ifthenelse{\equal{#3}{}}{}{\label{REQ:#3}\label{#3}} % je ein Label für Nr und Seite setzen
  \addReqToList{\thereq}{#1}                            % und in temp Req-Datei schreiben
}% end \requirement
%
%% -------- Neuen Sub-Requirement Eintrag erzeugen --------------------------
%
\renewcommand{\YOsubrequirement}[3]{
              \subrequirement{#1}{#2}{#3}}%             % Params = Header, Text, optional: Label
%
\newcommand{\subrequirement}[3]{%
  \refstepcounter{sorter}                               % mitzählen
  \refstepcounter{subreq}
  \index[Requirements]{\thereqsorter@\tmpIdxMain!\thesorter@\thesubreq~#1}% %hier den sorter
%    vor den Indexeintrag setzen, damit später richtig sortiert werden kann
  \YOprintSubReq {REQ~\thesubreq}{#1}{#2}
  \ifthenelse{\equal{#3}{}}{}{\label{REQ:#3}\label{#3}} % je ein Label für No und Seite setzen
  \addReqToList{\thesubreq}{#1}                         % und in temp Req-Datei schreiben
}% end \subrequirement
%
%% -------- Überschriftenfarben setzen ----------------------------
%%
\newcommand{\YOtitleColor}{black}%	
\titleformat{\section}%
 	{\color{\YOtitleColor}\normalfont\Large\bfseries}%
 	{\color{\YOtitleColor}\thesection}{1em}{}%
%
\titleformat{\subsection}%
 	{\color{\YOtitleColor}\normalfont\Large\bfseries}%
 	{\color{\YOtitleColor}\thesubsection}{1em}{}%
%
\titleformat{\subsubsection}%
 	{\color{\YOtitleColor}\normalfont\bfseries}%
 	{\color{\YOtitleColor}\thesubsubsection}{1em}{}%
%
\titleformat{\paragraph}%
 	{\color{\YOtitleColor}\normalfont\bfseries}%
 	{\color{\YOtitleColor}\theparagraph}{1em}{}%
%
\titleformat{\subparagraph}%
 	{\color{\YOtitleColor}\normalfont\bfseries}%
 	{\color{\YOtitleColor}\thesubparagraph}{1em}{}%
%
%----------------------------------------------------------------------------------------
%	TITLE PAGE
%----------------------------------------------------------------------------------------
%
\title{%
  \begin{figure}[!htp]%                          % set Yokogawa Logo on Title
    \center%                                     % page
      \includegraphics[width=0.50\columnwidth]{\YokogawaLogo}\\%
  \end{figure}%
  %
  \vspace{2in}%
  \textmd{\textbf{\YOTitle}}\\%
  \textmd{\textbf{\YOSubTitle}}\\%
  %
  \vspace{2in}%
  \vspace{0.1in}\normalsize{\textit{\YOVersion\ }}   \\%
  \vspace{0.1in}\normalsize{\textit{\YOAuthorName\ }} \\%
  %
  \date{\normalsize \YODate}%
}%
%
\pagenumbering{Alph}%
%
\SetWatermarkText{\YOWaterMark}%                %%%  put a watermark
\SetWatermarkScale{0.5}%                        %%%  on all pages.
\SetWatermarkColor{\YOWaterMarkColor}%
%
%
%----Lua Commands addReqToList ------------------------------------

\newcommand{\addReqToList}[2]{
\directlua{%
  if #1==1 then
    file = io.open("\YOreqfile", "w")
  else
    file = io.open("\YOreqfile", "a")
  end
  io.output(file)
  io.write("#1")
  io.write(string.char(38))
  io.write("#2")
  io.write(string.char(38))
  io.write("  ")
  io.write(string.char(92))
  io.write(string.char(92))
  io.write(string.char(13))
  io.close(file)
 }%
}% end addReqToList
%----------------------------------------------------------------------------------------
%
%----Lua Command readReqList ------------------------------------
\renewcommand{\YOprintReqCheckList}{\readReqList} % cmd to Print Requirements List Table
\newcommand{\reqTableLine}{ & & \\}
\newcommand{\readReqList}{
\newpage
%\LARGE{\textbf {Requirements Check List}}
\section {Requirements Check List}
\renewcommand*{\arraystretch}{2.4}
\begin{longtable}{| l | p{.60\textwidth} |p{.20\textwidth}|}
\hline
\rowcolor{yellow}  \textbf {REQ No.} & \textbf {Description} & \textbf {Comment} \\
\hline \endhead
\directlua{%
   file = io.open("\YOreqfile", "r")
   if not (file==nil) then
   for line in file:lines() do
      tex.print(line)
      tex.print([[\noexpand\hline]])
%     tex.print([[\noexpand\reqTableLine]])
    end
    io.close(file)
    end
 }% end directlua
\hline
\end{longtable}
}% end readReqList
%----------------------------------------------------------------------------------------
\YOdoLog {PARSING PREAMBLE FINISHED}%
