%\begin{comment}  % Beispiele
\YOdoLog{begin document "MyExamples"}
%----------------------------------------------------------------------------------------
%	Examples
%----------------------------------------------------------------------------------------
\pagebreak
\section*{Beispielsammlung für \LaTeX Anweisungen etc.}


Schriftarten:
{\fontsize{40}{48} \selectfont Text}

siehe \url {http://www.tug.dk/FontCatalogue/}

Usage

usepackage{aurical}
usepackage[T1]{fontenc}



%Style examples
%{
%    \Fontskrivan
%    \normalsize
%    Style example Fontskrivan  äöü ÄÖÜ ß  \blindtext
%}


{
    \ffmfamily
    Style example ffmfamily äöü ÄÖÜ ß  \blindtext
}

    fontsizes

    {\tiny 	    tiny       }
       	
    {\large     large}

    {\scriptsize 	 scriptsize}	

    {\Large         Large}

    {\footnotesize 	 	footnotesize}

    {\LARGE         LARGE}

    {\small 	 	small }

    {\huge          huge}

    {\normalsize 	normalsize} 	

    {\Huge          Huge}


{\fontsize{40}{48} \ffmfamily   und hier der Test in ffmfamily in  40 48}

%{\fontsize{40}{48} \Fontskrivan   und hier der Test in Fontskrivan in  40 48}



Zum Beispiel kann man mit \texttt{\textbackslash yokogawa} das Logo \yokogawa in den Text einfliessen lassen


mein eigener counter und subcounter.

Wenn der counter incrementiert (also bei jedem Aufruf) dann wird der Subcounter wieder zurückgesetzt (also wie section und subsection)


counter = \YOcounter
\begin{quote}
    subcounter = \YOsubcounter  \label{MyLabel1}

    subcounter = \YOsubcounter

    subcounter = \YOsubcounter

\end{quote}
counter = \YOcounter
\begin{quote}
    subcounter = \YOsubcounter

    subcounter = \YOsubcounter

    subcounter = \YOsubcounter   \label{MyLabel2}

\end{quote}

\YOsetcounter{5}
counter = \YOcounter  %% is 6(!) now. cause counter increments before print
\begin{quote}
    subcounter = \YOsubcounter

    subcounter = \YOsubcounter

    subcounter = \YOsubcounter

\end{quote}



Gallia est omnes divisa in partes tres quarum unum incolunt belagae, alium Aquitanae Gallia est omnes divisa in partes tres quarum unum incolunt belagae, alium AquitanaeGallia est omnes divisa in partes tres quarum unum incolunt belagae, alium AquitanaeGallia est omnes divisa in partes tres quarum unum incolunt belagae, alium AquitanaeGallia est omnes divisa in partes tres quarum unum incolunt belagae, alium AquitanaeGallia est omnes divisa in partes tres quarum unum incolunt belagae, alium AquitanaeGallia est omnes divisa in partes tres quarum unum incolunt belagae, alium AquitanaeGallia est omnes divisa in partes tres quarum unum incolunt belagae, alium AquitanaeGallia est omnes divisa in partes tres quarum unum incolunt belagae, alium AquitanaeGallia est omnes divisa in partes tres quarum unum incolunt belagae, alium AquitanaeGallia est omnes divisa in partes tres quarum unum incolunt belagae, alium AquitanaeGallia est omnes divisa in partes tres quarum unum incolunt belagae, alium AquitanaeGallia est omnes divisa in partes tres quarum unum incolunt belagae, alium AquitanaeGallia est omnes divisa in partes tres quarum unum incolunt belagae, alium AquitanaeGallia est omnes divisa in partes tres quarum unum incolunt belagae, alium Aquitanae

%--- Blind-Texte ---------------------------------------

    Eine Itemize-Umgebung setzen.
\blinditemize
    Eine Enumerate-Umgebung setzen.
\blindenumerate
    Eine Description-Umgebung setzen.
\blinddescription
    Eine Itemize-Umgebung mit je einem Absatz setzen.
\Blinditemize
    Eine Enumerate-Umgebung mit je einem Absatz setzen.
\Blindenumerate
    Eine Description-Umgebung mit je einem Absatz setzen.
\Blinddescription
    Eine Mathpaper erstellen

\blindmathpaper

%--- Zitat -----------------------------------------------

\begin{quote}
Ich zitiere:
\small
\it
\blindtext
\end{quote}

%---- Strukturierung ---------------------------------------------

\section{       lorem 1}  \label{PAGE_blabla}
\blindtext
\subsection{    ipsum 1.1}
\blindtext
\subsubsection{ quantum 1.1.1}
\blindtext
\paragraph{     librum 1.1.1.1}
\blindtext
\subparagraph{  maximum 1.1.1.1.1}
\blindtext

%---- insert Grafics (png, jpg, pdf ...)  -----------------------------------------------
\newpage

\section{Beispielgrafik}
Position Parameter:

h here: at the place in the text where the environment occurs.

H exactly here ..

t top: at the top of a text page.

b bottom: at the bottom of a text page.

p page: on a special page containing only floats

    \begin{figure}[htp]  %% position figure exactly "h"ere, else t or p
                         %% needs usepackage{float}
      \begin{center}
        \includegraphics[width=0.90\columnwidth]{\YokogawaLogo}
        \caption{Advanced Solutions Gateway (Layer Architecture)}
        \label{PIC_ADV_SOL_LAYERS}
      \end{center}
    \end{figure}

%---- insert Table ---------------------------------------------------------------------
\section{Tabellen}
\subsection{Beispieltabelle 1}
\small
\vspace{0.1in}
\begin{tabular}{|c|l|r|r|r|} \hline   % 5 Spalten (center, left, 3 x right justify)
    \textbf{Spalte 1} &
    \textbf{Spalte 2}









    &
    \textbf{Spalte 3} &
    \textbf{Spalte 4} &
    \textbf{Spalte 5} \\ \hline \hline

1. Zeile  &  1.2 &  1.3  & 1.4  & 1.5 \\
          &  noch immer 1. Zeile   &      &           &                 \\          \hline

2. Zeile  &  2.2 &  2.3  & 2.4  & 2.5 \\
          &  noch immer 2. Zeile   &      &           &                 \\ \hline   \hline

% letzte Zeile
	      & \textbf{Summe Phase 1} &	  &           &   \textbf{15.750} EUR    \\ \hline

 \end{tabular}
\normalsize

\subsection{Beispieltabelle 2}
\small
\begin{tabular}{|l|p{4cm}|}
\hline Dieser besonders lange Text bricht einfach nicht um! &
Dies ist ein langer, umbrechbarer Text der genau
so breit ist, wie er in der Definition oben definiert wurde.
Das bedeutet in diesem Beispiel also genau 4cm\\
\hline
\end{tabular}

\subsection{Beispieltabelle 3}

\begin{tabularx}{\textwidth}{|l|X|l|}
\hline Eine ganz normale dynamische Spalte & Diese Spalte dehnt sich& Eine normale Spalte\\ \hline
\end{tabularx}

\subsection{Beispieltabelle 4}

%\begin{tabularx}{8cm}{|X|X|X|X|}
%\hline
%In dieser Tabelle & hat jede Zelle genau die & gleich Breite & nämlich gerade
%2cm \\
%\hline
%Und wie man & dabei leicht erkennen kann & reicht diese Breite nicht bei allen
% & Spalten aus um den gesamten Text darzustellen. \\
%\hline
%\end{tabularx}

\subsection{Beispieltabelle mit Farbe}

\begin{tabularx}{\textwidth}{|X|l|} \hline
\rowcolor{yellow}  OS      & Features \\ \hline  \hline
iOS     & blah blah. \\  \hline
Android & blah blah. \\   \hline \hline
\rowcolor{green}  lastline      & no Features \\ \hline
\end{tabularx}


%---- Auflistung ------------------------------------------------------------------------

\section{Listen und Aufzählungen}
\subsection{Beispielliste}
\begin{itemize}
    \item alpha
    \item beta
    \item gamma
\end{itemize}


%---- Aufzählung mit Auflistung ---------------------------------------------------------

\subsection{Beispielaufzählung}

\begin{enumerate}

    \item eins
        \begin{itemize}
            \item 	alpha 1
            \item   beta 1
            \item 	gamma 1
        \end{itemize}

    \item zwei
        \begin{itemize}
            \item 	alpha 2
            \item   beta 2
            \item 	gamma 2
        \end{itemize}

    \item drei
        \begin{itemize}
            \item 	alpha 3
            \item   beta 3
            \item 	gamma 3
        \end{itemize}

\end{enumerate}

%---- div. Formats ----------------------------------------------------------------------
%übergeordnete Struktur  :  \part{hier gehts dann los}
%                           \part{oder auch hier}
draw a line \dots

\rule{\textwidth}{1pt} % draw a line

\textit{... italic Italic shape, used mostly for emphasis}

\textsl{... slanted Slanted shape, a bit different from italic}

\textsc{... Small Caps Small caps shape, use sparingly}

\textup{... upright Upright shape, usually the default}

\textbf{... boldface Boldface series, often used for headings}

\textmd{... medium Medium series, usually the default}

\textrm{... roman Roman family, usually the default}

\textsf{... sans serif Sans Serif family, used for posters, etc.}

\texttt{... typewriter for fix pitched characters (f.e. coding)}

\emph{... emphasized Use for emphasis, usually changes to italic}

\rule{\textwidth}{0.5pt} % draw a line
\dots end with a line

A random number:
\begin{luacode}
    tex.print(math.random())
\end{luacode}

the standard approximation $\pi = \directlua{tex.sprint(math.pi)}$


2-spaltiger Text        :  documentclass[12pt,twocolumn]{book}

\begin{verbatim}
Seitenwechsel           :  \pagebreak oder \newpage oder
                            \clearpage (verändern ggf. Seitenaufteilung)
Trennung definieren     :  In\-di\-vi\-du\-al\-lö\-sun\-gen
Trennung verhindern     :  "nicht\~trennen" = "nichttrennen" \\[2cm]
Absatz zusammenhalten   :  \begin{samepage} ... \end{samepage}

Einrückungen beginnen mit  \begin{quote} und enden mit \end{quote}
Fußnote                 :  \footnote{lorem ipsum}
Hyperlink               :  \href{https://de.wikipedia.org/}{lorem ipsum}

Label                   :  \label{LAB_blabla}
Referenz to Label       :  \ref{LAB_blabla}
Referenz to page        :  \pageref{PAGE_blabla}

\end{verbatim}
$3^{rd}$ Party Systeme  :  \verb+ $3^{rd}$ Party Systeme +

\section{my Requirements Specifications}



Auf dieser Seite sind die Begriffe System\index{System},
Systemzustand\index{System!Zustand}, Systemelement\index{System!Element} und
Emergenz\index{Emergenz} von Interesse.



Alpha 2\index{Alpha 2}
Alpha 3\index{Alpha 3}
Alpha 1\index{Alpha 1}
Alpha 11\index{Alpha 11}
Alpha 12\index{Alpha 12}

vorgezogene Referenz auf \reqref{ReqLblII} auf Seite \pageref{ReqLblII}

vorgezogene Referenz auf \reqref{ReqLblIII} auf Seite \pageref{ReqLblIII}

\requirement{Erstes äöü Requirement}{\blindtext}{ReqLblI}

\requirement{R Stufe 1}{}{}
\YOrequirement{YO Stufe 1}{}{}
\YOrequirement{R Stufe 1}{}{}
\YOrequirement{R Stufe 1}{}{}
\YOrequirement{R Stufe 1}{}{}
\YOrequirement{R Stufe 1}{}{}
\YOrequirement{R Stufe 1}{}{}
\YOrequirement{R Stufe 1}{}{}
\YOrequirement{R Stufe 1}{}{}
\YOrequirement{R Stufe 1}{}{}
\YOrequirement{R Stufe 1}{}{}

\YOsubrequirement{sub Requirement}{1 \blindtext}{sIReqLblI}{}

\YOsubrequirement{noch so ein sub Requirement }{ \blindtext

    SubReferenz auf \reqref{sIIIReqLblI} auf Seite \pageref{sIIIReqLblI}

    SubReferenz auf \reqref{sIIReqLblI} auf Seite \pageref{sIIReqLblI}

    SubReferenz auf \reqref{sIReqLblI} auf Seite \pageref{sIReqLblI}

}{sIIReqLblI}{} %% end of SubREQ 1.2

\subrequirement{Drittes sub Requirement}{\blindtext}{sIIIReqLblI}{}

\subrequirement{verweisendes Requirement}{
  in diesem REQ verweise ich mal auf \reqref{sIIIReqLblI} auf Seite \pageref{sIIIReqLblI} und SubReferenz auf \reqref{sIReqLblI} auf Seite \pageref{sIReqLblI}
}{ }{ }


\subrequirement{more sub     b Requirements ...}{ }{}
\subrequirement{more sub Requirements ...}{ }{}
\subrequirement{more sub Requirements ...}{ }{}
\subrequirement{more sub Requirements ...}{ }{}
\subrequirement{more sub Requirements ...}{ }{}
\subrequirement{more sub Requirements ...}{ }{}
\subrequirement{more sub Requirements ...}{ }{}
\subrequirement{more sub d Requirements ...}{ }{}
\subrequirement{more sub c Requirements ...}{ }{}
\subrequirement{more sub b Requirements ...}{ }{}
\subrequirement{more sub a Requirements ...}{ }{}
\subrequirement{more sub ab Requirements ...}{ }{}
\subrequirement{auch ein besonders langes  Requirements ...}{ }{}
\subrequirement{more sub Requirements ...}{ }{}

Referenz auf \YOreqref{ReqLblII} auf Seite \pageref{ReqLblII}

Referenz auf \reqref{ReqLblI} auf Seite \pageref{ReqLblI}

\requirement{Zweites Requirement}{\blindtext[2]}{ReqLblII}{}


\subrequirement{more sub aaa Requirements ...}{ }{}
\subrequirement{more sub abb Requirements ...}{ }{}
\subrequirement{more sub acc Requirements ...}{ }{}
\subrequirement{more sub Requirements die einer gründlichen Betrachtung bedürfen ...}{ }{}



Referenz auf \reqref{ReqLblII} auf Seite \pageref{ReqLblII}

 und

Referenz auf \reqref{ReqLblI} auf Seite \pageref{ReqLblI}


\requirement{Ein kompliziertes Requirement}{\blindtext}{ReqLblIII}{}

SubReferenz auf \reqref{sIIIReqLblI} auf Seite \pageref{sIIIReqLblI}

SubReferenz auf \reqref{sIReqLblI} auf Seite \pageref{sIReqLblI}

\subrequirement{more sub Requirements ...}{ }{}
\subrequirement{more sub Requirements ...}{ }{}
\subrequirement{more sub Requirements ...}{ }{}
\subrequirement{more sub Requirements ...}{ }{}
\subrequirement{more sub Requirements ...}{ }{}
\subrequirement{more sub Requirements ...}{ }{}
\subrequirement{more sub Requirements ...}{ }{}
\subrequirement{more sub Requirements ...}{ }{}
\subrequirement{more sub Requirements ...}{ }{}
\subrequirement{more sub Requirements ...}{ }{}
\subrequirement{more sub Requirements ...}{ }{}


\YOdoLog{end document "MyExamples"}
%\end{comment} 